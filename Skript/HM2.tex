\documentclass[11pt,a4paper]{book} 
\usepackage[utf8]{inputenc}
\usepackage[ngerman]{babel}
\usepackage{amsmath}
\usepackage{amssymb}
\usepackage{bbm}
\usepackage{geometry}
\usepackage{graphicx}
\usepackage{verbatim}
\geometry{
  left=3cm,
  right=3cm,
  top=3cm,
  bottom=3cm,
  bindingoffset=5mm
}
\usepackage{enumerate}
\usepackage[T1]{fontenc} 
\usepackage{ucs}
\usepackage[utf8]{inputenc}

\newcommand {\Q}	{\mathbb{Q}}
\newcommand {\R}	{\mathbb{R}}
\newcommand {\C}	{\mathbb{C}}
\newcommand {\Rn}	{\mathbb{R}^n}
\newcommand {\Rzwei}	{\mathbb{R}^2}
\newcommand {\Rnxn}	{\mathbb{R}^{n \times n}}
\newcommand {\Rmxn}	{\mathbb{R}^{m \times n}}
\newcommand {\Rmxm}	{\mathbb{R}^{m \times m}}
\newcommand {\Cn}	{\mathbb{C}^n}
\newcommand {\Cnxn}	{\mathbb{C}^{n \times n}}
\newcommand {\N}	{\mathbb{N}}
\newcommand{\1}    	{\mathbbm{1}}
\newcommand{\Onot}		{\mathcal{O}}
\newcommand{\diag}	{\textrm{diag}}

\begin{document}

\tableofcontents 

\chapter*{Einführung}
\section{Stetigkeit in einer Dimension}
\begin{align*}
	f \textrm{ ist stetig in } x_0 & \\
	&\quad \Leftrightarrow \quad
	\lim_{x \rightarrow x_0} f(x) = f(x_0) \\
	&\quad \Leftrightarrow \quad
	\forall \left(x_n\right) \textrm{ mit } \lim_{n \rightarrow \infty} x_n = x_0
	\textrm{ gilt }  \lim_{n \rightarrow \infty} f(x_n) = f(x_0) \\
	&\quad \Leftrightarrow \quad \forall~ \varepsilon > 0 \quad \exists~ \delta \quad \textrm{mit} \quad \vert f(x) - f(x_0) \vert < \varepsilon \quad \forall~ x \in \left( x_0 - \delta, x_0 + \delta \right)
\end{align*}
\textit{Bemerkung}: Der Grenzwert von Funktionen ist über den Grenzwert von Folgen definiert und kann auch nur so überprüft werden.

\section{Zwei Sonderfälle}
\subsection*{Skalarfeld}
Sei \( f : \Rzwei \rightarrow \R \) \\
Visualisierung durch Höhenlinien: \( H_c := \left\{ x \in \Rn : f(x) = c \right\} \) \\
Beispiel: \( f(x,y) = x^2 + y^2 \)

\subsection*{Vektorfeld}
Sei \( f : \Rzwei \rightarrow \Rzwei \) \\
Beispiel: \(f(x,y) = \left(\begin{array}{c} x \\ y \end{array} \right)\)

\chapter{Differentialrechnung in höheren Dimensionen}

\section{}

\subsubsection*{Skalarprodukt}
Definition: \( \left< x,y \right> := x^\top y = \sum_{k=1}^n x_k y_k \) für \(x,y \in \Rn\)

\subsubsection*{Euklidische Norm}
Definition: \( \Vert x \Vert_2 := \sqrt{\left< x,x \right>} = \sqrt{\sum_{k=1}^n x_k^2} \)

\subsection{Korollar}
Sei \(x \in \Rn\) mit \(x = \left(\begin{array}{c} x_1 \\ \vdots \\ x_n \end{array}\right)\)
\begin{enumerate}[1.~]
	\item \begin{align*}
		\max_{1 \leqslant k \leqslant n} \vert x_k \vert \leqslant \Vert x \Vert \leqslant \sqrt{n} \max_{1 \leqslant k \leqslant n} \vert x_k \vert
	\end{align*}
	\item Cauchy-Schwarz-Ungleichung:
	\begin{align*}
		\forall~ x,y \in \Rn \quad : \quad \vert \left< x,y \right> \vert \leqslant \Vert x \Vert \cdot \Vert y \Vert
	\end{align*}
	Begründung (nicht Beweis!) durch alternative Definition: \( \left< x,y \right> = \Vert x \Vert \cdot \Vert y \Vert \underbrace{\cos \alpha}_{\leqslant 1} \) \\
	Dabei ist \(\alpha\) der Winkel der zwischen \(x\) und \(y\) eingeschlossen wird. \\
	Daraus folgt:
	\begin{align*}
		\vert \left< x,y \right> \vert = \Vert x \Vert \cdot \Vert y \Vert
		\Leftrightarrow
		x,y \textrm{ sind lin. unabhängig} : x = \lambda y \textrm{ oder } y = \lambda x \textrm{ für } \lambda \in \R
	\end{align*}
	\item \(\Vert \cdot \Vert\) ist eine Norm. Eine Norm hat folgende Eigenschaften:
	\begin{enumerate}[(i)]
		\item \( \Vert x \Vert \geqslant 0 \) und \( \Vert x \Vert = 0 \Leftrightarrow x = 0 \)
		\item \( \Vert \lambda x \Vert = \vert \lambda \vert \cdot \Vert x \Vert  \)
		\item \( \Vert x + y \Vert \leqslant \Vert x \Vert  + \Vert y \Vert \) Dreiecksungleichung
	\end{enumerate}
\end{enumerate}

\subsection{Konvention}
Für \(A \subset \Rn\) gilt für das Komplement \(A^c = \Rn \setminus A\)

\subsection{Definition der \(\varepsilon\)-Umgebung}
Sei \(x_0 \in \Rn\) und \(\varepsilon > 0\), dann gilt für die \(\varepsilon\)-Umgebung \(U_\varepsilon(x_0)\) von \(x_0\):
\begin{align*}
	U_\varepsilon(x_0) := \left\{ x \in \Rn : \Vert x - x_0 \Vert < \varepsilon \right\}
\end{align*}

\subsection{Topologische Grundbegriffe}
Sei \(A \subset \Rn\), dann heißt ein Punkt \(x_0 \in \Rn\)
\begin{enumerate}[(i)]
	\item ein \textbf{innerer Punkt}, wenn gilt \(\exists~ \varepsilon > 0\) mit \(U_\varepsilon(x_0) \subset A\) \\
	Menge aller inneren Punkte: \( \mathring{A} = \left\{ x \in \Rn : \exists~ \varepsilon > 0 \textrm{ mit } U_\varepsilon(x) \subset A \right\} \)
	\item ein \textbf{Berührungspunkt}, wenn \(\forall~ \varepsilon > 0\) gilt \(U_\varepsilon (x_0) \cap A \neq \varnothing \) \\
	\textbf{abgeschlossene Hülle}: \(\overline{A} = \left\{ x \in \Rn : \forall~ \varepsilon > 0 \textrm{ gilt } U_\varepsilon(x_0) \neq \varnothing \right\} \)
	\item ein \textbf{Häufungspunkt}, wenn \(\forall~ \varepsilon > 0\) gilt \( \left( U_\varepsilon(x_0) \setminus \left\{ x_0 \right\} \right) \cap A \neq \varnothing \) \\
	Die Menge aller Häufungspunkte wird mit \(A'\) bezeichnet.
	\item ein \textbf{Randpunkt}, wenn \(\forall~ \varepsilon > 0\) gilt \( U_\varepsilon(x_0) \cap A \neq \varnothing\) und \( U_\varepsilon(x_0) \cap A^c \neq \varnothing\) \\
	Menge aller Randpunkte oder auch \textbf{Rand} von \(A\) wird mit \(\partial A \) bezeichnet.
\end{enumerate}
\subsubsection*{Korollar}
\begin{enumerate}[(i)]
	\item \(\mathring{A} \subset A\)
	\item \(\mathring{A} \subset \overline{A}\)
	\item \(\partial A \subset \overline{A}\)
	\item \(\overline{A} = \mathring{A} \cup \partial A \)
	\item \(\overline{A} = A \cup \partial A \) (schwächere Aussage als (iv))
\end{enumerate}

\subsection{Definition}
Eine Menge \(A \subset \Rn\) heißt
\begin{enumerate}[(i)]
	\item \textbf{offen}, wenn \(A = \mathring{A} \) gilt (\(A\) besteht nur aus inneren Punkten)
	\item \textbf{abgeschlossen}, wenn \(\partial A \subset A \) gilt (wenn der Rand in der Menge enthalten ist)
\end{enumerate}

\subsection{Beispiele}
\begin{enumerate}[1.~]
	\item Jede \(\varepsilon\)-Umgebung \(U_\varepsilon(x_0 \in \Rn)\) ist offen
	\item Sei \(I \subset \R\), dann gilt
	\begin{enumerate}[(i)]
		\item \(I\) ist offen, wenn \(I = (a,b)\) mit \( -\infty \leqslant a \leqslant b \leqslant \infty \) \\
		für \(a = b\) gilt \(I = \varnothing\) mit \(I\) offen \\
		und für \(a = -\infty, b = \infty\) ist \(I\) auch offen
		\item \(I\) ist abgeschlossen, wenn \(I = [a,b]\) mit \(a,b \in \R\) \\
		oder \(I = (-\infty, b]\) oder \(I = [a, \infty) \) oder \(I = (-\infty, \infty) = \R\)
	\end{enumerate}
	(die reellen Zahlen sind offen und abgeschlossen zugleich)
\end{enumerate}

\subsection{Satz}
für \(A \subset \Rn\) sind folgenden Aussagen äquivalent:
\begin{enumerate}[(i)]
	\item \(A\) ist abgeschlossen \(A = \overline{A}\)
	\item \(A\) enthält alle Häufungspunkte, \(A' \subset A\)
	\item \(A\) enthält alle Randpunkte, \(\partial A \subset A\)
	\item \(A^c\) ist offen
\end{enumerate}

\subsection{Satz}
\begin{enumerate}[(i)]
	\item \(\varnothing\) und \(\Rn\) sind offen
	\item Die Vereinigung beliebig vieler offene Mengen \(O_j\) mit \(j \in J\) ist stets offen
	\item Der Durchschnitt \underline{endlich} vieler offener Mengen \(O_1, ..., O_r\) ist stets offen
\end{enumerate}

\end{document}