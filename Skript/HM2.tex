\documentclass[11pt,a4paper]{book} 
\usepackage[utf8]{inputenc}
\usepackage[ngerman]{babel}
\usepackage{amsmath}
\usepackage{amssymb}
\usepackage{bbm}
\usepackage{geometry}
\usepackage{graphicx}
\usepackage{verbatim}
\geometry{
  left=3cm,
  right=3cm,
  top=3cm,
  bottom=3cm,
  bindingoffset=5mm
}
\usepackage{enumerate}
\usepackage[T1]{fontenc} 
\usepackage{ucs}
\usepackage[utf8]{inputenc}

\newcommand {\Q}	{\mathbb{Q}}
\newcommand {\R}	{\mathbb{R}}
\newcommand {\C}	{\mathbb{C}}
\newcommand {\Rn}	{\mathbb{R}^n}
\newcommand {\Rm}	{\mathbb{R}^m}
\newcommand {\Rzwei}	{\mathbb{R}^2}
\newcommand {\Rnxn}	{\mathbb{R}^{n \times n}}
\newcommand {\Rmxn}	{\mathbb{R}^{m \times n}}
\newcommand {\Rmxm}	{\mathbb{R}^{m \times m}}
\newcommand {\Cn}	{\mathbb{C}^n}
\newcommand {\Cnxn}	{\mathbb{C}^{n \times n}}
\newcommand {\N}	{\mathbb{N}}
\newcommand{\1}    	{\mathbbm{1}}
\newcommand{\Onot}		{\mathcal{O}}
\newcommand{\diag}	{\textrm{diag}}
\newcommand{\mitt}	{\textrm{ mit }}

\begin{document}

\tableofcontents 

\chapter*{Einführung}
\section{Stetigkeit in einer Dimension}
\begin{align*}
	f \textrm{ ist stetig in } x_0 & \\
	&\quad \Leftrightarrow \quad
	\lim_{x \rightarrow x_0} f(x) = f(x_0) \\
	&\quad \Leftrightarrow \quad
	\forall \left(x_n\right) \textrm{ mit } \lim_{n \rightarrow \infty} x_n = x_0
	\textrm{ gilt }  \lim_{n \rightarrow \infty} f(x_n) = f(x_0) \\
	&\quad \Leftrightarrow \quad \forall~ \varepsilon > 0 \quad \exists~ \delta \quad \textrm{mit} \quad \vert f(x) - f(x_0) \vert < \varepsilon \quad \forall~ x \in \left( x_0 - \delta, x_0 + \delta \right)
\end{align*}
\textit{Bemerkung}: Der Grenzwert von Funktionen ist über den Grenzwert von Folgen definiert und kann auch nur so überprüft werden.

\section{Zwei Sonderfälle}
\subsection*{Skalarfeld}
Sei \( f : \Rzwei \rightarrow \R \) \\
Visualisierung durch Höhenlinien: \( H_c := \left\{ x \in \Rn : f(x) = c \right\} \) \\
Beispiel: \( f(x,y) = x^2 + y^2 \)

\subsection*{Vektorfeld}
Sei \( f : \Rzwei \rightarrow \Rzwei \) \\
Beispiel: \(f(x,y) = \left(\begin{array}{c} x \\ y \end{array} \right)\)

\chapter{Differentialrechnung in höheren Dimensionen}

\section{Topologie}

\subsubsection*{Skalarprodukt}
Definition: \( \left< x,y \right> := x^\top y = \sum_{k=1}^n x_k y_k \) für \(x,y \in \Rn\)

\subsubsection*{Euklidische Norm}
Definition: \( \Vert x \Vert_2 := \sqrt{\left< x,x \right>} = \sqrt{\sum_{k=1}^n x_k^2} \)

\subsection{Korollar}
Sei \(x \in \Rn\) mit \(x = \left(\begin{array}{c} x_1 \\ \vdots \\ x_n \end{array}\right)\)
\begin{enumerate}[1.~]
	\item \begin{align*}
		\max_{1 \leqslant k \leqslant n} \vert x_k \vert \leqslant \Vert x \Vert \leqslant \sqrt{n} \max_{1 \leqslant k \leqslant n} \vert x_k \vert
	\end{align*}
	\item Cauchy-Schwarz-Ungleichung:
	\begin{align*}
		\forall~ x,y \in \Rn \quad : \quad \vert \left< x,y \right> \vert \leqslant \Vert x \Vert \cdot \Vert y \Vert
	\end{align*}
	Begründung (nicht Beweis!) durch alternative Definition: \( \left< x,y \right> = \Vert x \Vert \cdot \Vert y \Vert \underbrace{\cos \alpha}_{\leqslant 1} \) \\
	Dabei ist \(\alpha\) der Winkel der zwischen \(x\) und \(y\) eingeschlossen wird. \\
	Daraus folgt:
	\begin{align*}
		\vert \left< x,y \right> \vert = \Vert x \Vert \cdot \Vert y \Vert
		\Leftrightarrow
		x,y \textrm{ sind lin. unabhängig} : x = \lambda y \textrm{ oder } y = \lambda x \textrm{ für } \lambda \in \R
	\end{align*}
	\item \(\Vert \cdot \Vert\) ist eine Norm. Eine Norm hat folgende Eigenschaften:
	\begin{enumerate}[(i)]
		\item \( \Vert x \Vert \geqslant 0 \) und \( \Vert x \Vert = 0 \Leftrightarrow x = 0 \)
		\item \( \Vert \lambda x \Vert = \vert \lambda \vert \cdot \Vert x \Vert  \)
		\item \( \Vert x + y \Vert \leqslant \Vert x \Vert  + \Vert y \Vert \) Dreiecksungleichung
	\end{enumerate}
\end{enumerate}

\subsection{Konvention}
Für \(A \subset \Rn\) gilt für das Komplement \(A^c = \Rn \setminus A\)

\subsection{Definition der \(\varepsilon\)-Umgebung}
Sei \(x_0 \in \Rn\) und \(\varepsilon > 0\), dann gilt für die \(\varepsilon\)-Umgebung \(U_\varepsilon(x_0)\) von \(x_0\):
\begin{align*}
	U_\varepsilon(x_0) := \left\{ x \in \Rn : \Vert x - x_0 \Vert < \varepsilon \right\}
\end{align*}
\textit{Bemerkung}: Die punktierte \(\varepsilon\)-Umgebung ist definiert als: \( \dot{U}_\varepsilon = U_\varepsilon (a) \setminus \left\{ a \right\} \)

\subsection{Topologische Grundbegriffe}
Sei \(A \subset \Rn\), dann heißt ein Punkt \(x_0 \in \Rn\)
\begin{enumerate}[(i)]
	\item ein \textbf{innerer Punkt}, wenn gilt \(\exists~ \varepsilon > 0\) mit \(U_\varepsilon(x_0) \subset A\) \\
	Menge aller inneren Punkte: \( \mathring{A} = \left\{ x \in \Rn : \exists~ \varepsilon > 0 \textrm{ mit } U_\varepsilon(x) \subset A \right\} \)
	\item ein \textbf{Berührungspunkt}, wenn \(\forall~ \varepsilon > 0\) gilt \(U_\varepsilon (x_0) \cap A \neq \varnothing \) \\
	\textbf{abgeschlossene Hülle}: \(\overline{A} = \left\{ x \in \Rn : \forall~ \varepsilon > 0 \textrm{ gilt } U_\varepsilon(x_0) \neq \varnothing \right\} \)
	\item ein \textbf{Häufungspunkt}, wenn \(\forall~ \varepsilon > 0\) gilt \( \left( U_\varepsilon(x_0) \setminus \left\{ x_0 \right\} \right) \cap A \neq \varnothing \) \\
	Die Menge aller Häufungspunkte wird mit \(A'\) bezeichnet.
	\item ein \textbf{Randpunkt}, wenn \(\forall~ \varepsilon > 0\) gilt \( U_\varepsilon(x_0) \cap A \neq \varnothing\) und \( U_\varepsilon(x_0) \cap A^c \neq \varnothing\) \\
	Menge aller Randpunkte oder auch \textbf{Rand} von \(A\) wird mit \(\partial A \) bezeichnet.
\end{enumerate}
\subsubsection*{Korollar}
\begin{enumerate}[(i)]
	\item \(\mathring{A} \subset A\)
	\item \(\mathring{A} \subset \overline{A}\)
	\item \(\partial A \subset \overline{A}\)
	\item \(\overline{A} = \mathring{A} \cup \partial A \)
	\item \(\overline{A} = A \cup \partial A \) (schwächere Aussage als (iv))
\end{enumerate}

\subsection{Definition von offen und abgeschlossen}
Eine Menge \(A \subset \Rn\) heißt
\begin{enumerate}[(i)]
	\item \textbf{offen}, wenn \(A = \mathring{A} \) gilt (\(A\) besteht nur aus inneren Punkten)
	\item \textbf{abgeschlossen}, wenn \(\partial A \subset A \) gilt (wenn der Rand in der Menge enthalten ist)
\end{enumerate}

\subsection{Beispiele}
\begin{enumerate}[1.~]
	\item Jede \(\varepsilon\)-Umgebung \(U_\varepsilon(x_0 \in \Rn)\) ist offen
	\item Sei \(I \subset \R\), dann gilt
	\begin{enumerate}[(i)]
		\item \(I\) ist offen, wenn \(I = (a,b)\) mit \( -\infty \leqslant a \leqslant b \leqslant \infty \) \\
		für \(a = b\) gilt \(I = \varnothing\) mit \(I\) offen \\
		und für \(a = -\infty, b = \infty\) ist \(I\) auch offen
		\item \(I\) ist abgeschlossen, wenn \(I = [a,b]\) mit \(a,b \in \R\) \\
		oder \(I = (-\infty, b]\) oder \(I = [a, \infty) \) oder \(I = (-\infty, \infty) = \R\)
	\end{enumerate}
	(die reellen Zahlen sind offen und abgeschlossen zugleich)
\end{enumerate}

\subsection{Satz}
für \(A \subset \Rn\) sind folgenden Aussagen äquivalent:
\begin{enumerate}[(i)]
	\item \(A\) ist abgeschlossen \(A = \overline{A}\)
	\item \(A\) enthält alle Häufungspunkte, \(A' \subset A\)
	\item \(A\) enthält alle Randpunkte, \(\partial A \subset A\)
	\item \(A^c\) ist offen
\end{enumerate}

\subsection{Satz}
\begin{enumerate}[(i)]
	\item \(\varnothing\) und \(\Rn\) sind offen.
	\item Die Vereinigung beliebig vieler offene Mengen ist offen:
	\begin{align*}
		\bigcup_{j \in J} \left( O_j \textrm{ offen} \right) = O \textrm{ offen}
	\end{align*}
	\item Der Durchschnitt \underline{endlich} vieler offener Mengen ist offen:
	\begin{align*}
		\bigcap_{j = 1}^{n} \left( O_j \textrm{ offen} \right) = O \textrm{ offen}
	\end{align*}
	\textit{Bemerkung}: Für unendlich viele offene Mengen gilt dies nicht immer:
	\begin{align*}
		\bigcap_{k = 1}^{\infty} \left( -\frac{1}{k}, \frac{1}{k} \right) = \left( -1, 1 \right) \cap \left( -\frac{1}{2}, \frac{1}{2} \right) \cap \left( -\frac{1}{3}, \frac{1}{3} \right) \cap ... = \left\{ 0 \right\} \textrm{ abgeschlossen}
	\end{align*}
\end{enumerate}

\subsubsection{Beispiel}
Seien \(A_1, A_2\) zwei abgeschlossene Mengen, dann gilt
\begin{enumerate}[(i)]
	\item \(A_1 \cup A_2\) ist abgeschlossen \\
	\textit{Beweisidee}: \(A_1\) ist abgeschlossen \(\Rightarrow A_1^c\) ist offen \\
	\begin{align*}
		\left( A_1 \cup A_2 \right)^c \stackrel{\textrm{De Morgan}}{=}& \underbrace{A_1^c}_{\textrm{offen}} \cap \underbrace{A_2^c}_{\textrm{offen}} \textrm{ ist offen wegen Satz 1.1.8} \\
		\left( \left( A_1 \cup A_2 \right)^c \right)^c =& ~A_1 \cup A_2 \textrm{ ist abgeschlossen}
	\end{align*}
\end{enumerate}

\subsection{Satz}
\begin{enumerate}[(i)]
	\item \(\varnothing\) und \(\Rn\) sind abgeschlossen.
	\item Der Durchschnitt beliebig vieler abgeschlossener Mengen ist abgeschlossen:
	\begin{align*}
		\bigcap_{j \in J} \left( A_j \textrm{ abgeschlossen} \right) = A \textrm{ abgeschlossen}
	\end{align*}
	\item Die Vereinigung \underline{endlich} vieler abgeschlossenen Mengen ist abgeschlossen:
	\begin{align*}
		\bigcup_{j = 1}^{n} \left( A_j \textrm{ abgeschlossen} \right) = A \textrm{ abgeschlossen}
	\end{align*}
	\textit{Bemerkung}: Für unendlich viele abgeschlossene Mengen gilt dies nicht immer:
	\begin{align*}
		\bigcup_{k = 1}^{\infty} \left[ -1 + \frac{1}{n}, 1 - \frac{1}{n} \right] = \left\{ 0 \right\} \cup \left[ -\frac{1}{2}, \frac{1}{2} \right] \cup \left[ -\frac{2}{3}, \frac{2}{3} \right] \cup ... = \left( -1, 1 \right) \textrm{ offen}
	\end{align*}	 
\end{enumerate}

\subsection{Definition von beschränkt und kompakt}
Eine Menge \(A \subset \Rn\) heißt:
\begin{enumerate}[(i)]
	\item \textbf{beschränkt} wenn \(\exists~ c > 0 \) mit \( \Vert x \Vert < c \quad \forall~ x \in A\)
	\item \textbf{kompakt}, wenn \(A\) abgeschlossen und beschränkt ist.
\end{enumerate}

\section{Folgen}

\subsection{Definition von Konvergenz und Beschränktheit}

Eine Folge \((a_k)_{k=1}^\infty \) heißt
\begin{enumerate}[(i)]
	\item \textbf{konvergent}, wenn gilt
	\begin{align*}
		\exists~ a \in \Rn \quad \textrm{mit} \quad \forall~ \varepsilon > 0 \quad \exists~ N(\varepsilon) : \quad \Vert a_k - a \Vert \quad \forall~ k \geqslant N(\varepsilon)
	\end{align*}
	Dann ist \(a\) der Grenzwert der Folge:
	\begin{align*}
		a = \lim_{k \rightarrow \infty} a_k \quad \textrm{oder} \quad a_k \stackrel{k \rightarrow \infty}{\rightarrow} a
	\end{align*}
	\item \textbf{beschränkt}, wenn \(\exists~ c > 0 \) mit \(\Vert a_k \Vert < c \quad \forall~ k \)
\end{enumerate}

\subsection{Bemerkung}

Wenn eine Folge \((a_k) = \left( \begin{array}{c} \left( a_1^{(k)} \right) \\ \vdots \\ \left( a_n^{(k)} \right)	\end{array} \right) \in \Rn\) konvergiert, so gilt
\begin{enumerate}[(i)]
	\item  \(\Leftrightarrow\) jede Komponente \( \left( a_1^{(k)}\right), ..., \left( a_n^{(k)} \right) \) konvergiert:
	\begin{align*}
		\lim_{k \rightarrow \infty} a_k = a \quad \Leftrightarrow \quad \lim_{k \rightarrow \infty} a_i^{(k)} = a_i \quad \textrm{für } i = 1, ..., n
	\end{align*}
	\item \(\Leftrightarrow (a_k) \) erfüllt das \textbf{Cauchy-Kriterium}:
	\begin{align*}
		\forall~ \varepsilon > 0 \quad \exists~ N(\varepsilon) : \quad \Vert a_k - a_l \Vert < \varepsilon \quad \forall~ k,l \geqslant N(\varepsilon)
	\end{align*}
	\item \(\Leftrightarrow\) jede Teilfolge von \((a_k)\) konvergiert gegen \(a\): \( a_{l_k} \stackrel{k \rightarrow \infty}{\rightarrow} a \) für \( l_1 \geqslant 1, l_2 \geqslant 2, ...\)
	\item der Grenzwert \(a\) ist eindeutig.
\end{enumerate}

\subsection{Satz von Bolzano Weierstraß}
Jede beschränkte Folge im \(\Rn\) besitzt einen konvergente Teilfolge.

\subsubsection*{Beispiele}
\begin{enumerate}[(i)]
	\item \(n = 1\): Sei \( A \leqslant (a_k) \leqslant B \quad \forall~ k \). Konstruiert man eine neue Schranke mit \(\frac{A + B}{2} \) so liegen wiederum \(\infty\) viele Elemente in der oberen und/oder unteren Hälfte.
	\item Sei \((a_k) = \left( \begin{array}{c} (x_k) \\ (y_k) \end{array} \right) \) eine beschränkte Folge im \(\R^2\) \\
\(\Rightarrow\) \((x_k), (y_k)\) sind beschränkte Folgen \\
\(\stackrel{\substack{\textrm{Satz von}\\\textrm{Bolzano}\\\textrm{Wierstraß}}}{\Rightarrow} \exists~ (x_k), (y_k)\) sind konvergent
\end{enumerate}

\subsection{Abschließende Bemerkungen}
\begin{enumerate}[(i)]
	\item Grenzwert Rechenregeln können aus dem \(\R\) für \(\Rn\) übernommen werden. \\
	\textit{z.b.} \( a_k \stackrel{k \rightarrow \infty}{\rightarrow} a, \quad b_k \stackrel{k \rightarrow \infty}{\rightarrow} b 
	\quad \Rightarrow \quad a_k^\top b_k \stackrel{k \rightarrow \infty}{\rightarrow} a^\top b \)
	\item Es gibt viele Zusammenhänge zwischen den Eigenschaften von Folgen und den topologischen Eigenschaften von Mengen. \\
	\textit{z.b.} Sei \(A \subset \Rn\) und \(a \in \Rn\) ein Häufungspunkt \\
	\(\Leftrightarrow \quad \exists (a_k)_{k=1}^\infty \) mit \( a_k \in A \setminus \left\{ a \right\} \forall~ k \quad \) und \( \quad a_k \stackrel{k \rightarrow \infty}{\rightarrow} a\)
\end{enumerate}

\section{Funktionsgrenzwerte und Stetigkeit}

\subsection{Definition}
Eine Funktion \(f : A \subset \Rn \rightarrow \Rm \) nennt man eine Funktion mit \(n\)-Veränderlichen.
\begin{align*}
	f(x_1, ..., x_n) = f( \left( \begin{array}{c} x_1 \\ \vdots \\ x_n \end{array} \right) ) = \left( \begin{array}{c}
		f_1(x_1, ..., x_n) \\
		\vdots \\
		f_m(x_1, ..., x_n) \\
	\end{array} \right) \quad \textrm{mit} \quad f_1, ..., f_m : \Rn \rightarrow \R
\end{align*}

\subsection{Definition Grenzwert/Limes}
Sei \(f : A \subset \Rn \rightarrow \Rm \) und \(a \in \overline{A}\). Ein \(b \in \Rm\) heißt Grenzwert von \(f\) für \(x \rightarrow a\), wenn gilt:
\begin{align*}
	\forall \varepsilon > 0 \quad \exists~ \delta(\varepsilon) > 0 : \quad \Vert f(x) - b \Vert < \varepsilon \quad \forall~ x \in \dot{U}_{\delta(\varepsilon)}(a) \cap A
\end{align*}
\textit{Bemerkung}: Die Funktion \(f\) muss in \(a\) nicht stetig sein, so kann z.b. gelten: \\
\( \lim_{x \rightarrow a} f(x) = b \neq f(a) \)

\subsection{Bemerkung}
Sei \(f : A \subset \Rn \rightarrow \Rm, a \in \overline{A}, b \in \Rm\) dann sind folgende Aussagen äquivalent:
\begin{enumerate}[(i)]
	\item \( f(x) \stackrel{x \rightarrow a} b \)
	\item \( \Vert f(x) - b \Vert \stackrel{x \rightarrow a} 0 \in \R^1 \) (Eine Norm bildet immer auf ein Skalar ab)
	\item \( f_1(x) \stackrel{x \rightarrow a}{\rightarrow} b_1, ..., f_m(x) \stackrel{x \rightarrow a}{\rightarrow} b_m \)
\end{enumerate}
Zusätzlich gilt das \textbf{Cauchy-Kriterium}:
\begin{align*}
	\lim_{x \rightarrow a} f(x) = b \quad \Leftrightarrow \quad \forall~ \varepsilon > 0 ~ \exists~ \delta(\varepsilon) > 0: \quad 
	\Vert f(x), f(y) \Vert < \varepsilon \quad \forall~ x,y \in \dot{U}_{\delta(\varepsilon)}(a) \cap A
\end{align*}

\subsection{Beispiel}
Sei \( f(x,y) = \frac{xy}{x^2 + y^2} \)
\begin{align*}
	a_k &= \left( \begin{array}{c} x_k \\ y_k \end{array} \right) = \left( \begin{array}{c} \frac{1}{k} \\ \frac{1}{k} \end{array} \right), \quad 
	f(a_k) = \frac{\frac{1}{k^2}}{\frac{1}{k^2} + \frac{1}{k^2}} = \frac{1}{2} \quad \forall~ k \\
	b_k &= \left( \begin{array}{c} x_k \\ 0 \end{array} \right) \mitt x_k \stackrel{k \rightarrow \infty}{\rightarrow} 0, \quad
	f(b_k) = \frac{0}{x_k^2} \quad \forall~ k \\
	&\textrm{Da } \lim_{k \rightarrow \infty} f(a_k) = \frac{1}{2} \neq 0 = \lim_{k \rightarrow \infty} f(b_k) \textrm{ kann der Grenzwert nicht existieren.}
\end{align*}

\subsection{Lemma Folgenkriterium}
Sei \(f : A \subset \Rn \rightarrow \Rm, a \in \overline{A} \)
\begin{align*}
	\underbrace{\exists b \in \Rm \mitt \lim_{x \rightarrow a} f(x) = b}_{\textrm{der Grenzwert } b \textrm{ existiert}} \quad &\Leftrightarrow \quad
	\underbrace{
		\begin{array}{l}
			\textrm{jede Folge } (x_k)_{k=1}^\infty \subset A \mitt x_k \neq a ~ \forall~ k \textrm{ und } x_k \stackrel{k \rightarrow \infty}{\rightarrow} a \\
			\Rightarrow  \quad f(x_k) \stackrel{k \rightarrow \infty}{\rightarrow} b
		\end{array}
	}_{\textrm{jede beliebige Folge konvergiert gegen } b}
\end{align*}

\subsection{Satz zu Grenzwerte verketteter Funktionen}
Sei \(A \subset \Rn, B \subset \Rm, a \in \overline{A}, f: A \rightarrow B, g: \overline{B} \rightarrow \R^l \)
\begin{align*}
	\exists~ b \in \overline{B} \mitt \lim_{x \rightarrow a} f(x) = b, \quad
	\exists~ c \in \R^l \mitt \lim_{y \rightarrow b} g(y) = c \quad \Rightarrow \quad
	\lim_{x \rightarrow a} \underbrace{g\left(f(x)\right)}_{(g \circ f)(x)} = \lim_{y \rightarrow b} g(y) = c
\end{align*}

\subsection{Beispiel}
Sei \(f(x,y) = e^{-x^2 + y^2} = \exp\left( g(x,y) \right) \mitt g(x,y) = x^2 + y^2\), dabei gilt:
\begin{align*}
	\lim_{(x,y)^\top \rightarrow (0,0)^\top} g(x,y) = \lim_{(x,y)^\top \rightarrow (0,0)^\top} x^2 + y^2 = 0
	\quad \Rightarrow \quad \lim_{z \rightarrow 0} f(z) = \lim_{z \rightarrow 0} e^z = 1 \\
\end{align*}

\subsection{Definition der Stetigkeit}
Sei \(f: A \subset \Rn \rightarrow \Rm\)
\begin{enumerate}[(i)]
	\item \(f\) ist \textbf{stetig} in \(a \in A\) wenn gilt:
	\begin{align*}
		\forall~ \varepsilon > 0 ~ \exists \delta(\varepsilon) : \quad \Vert f(x) - f(a) \Vert < \varepsilon \quad \forall~ x \in U_{\delta(\varepsilon)}(a) \cap A
	\end{align*}
	\textit{Bemerkung}: Es wird \( \lim_{x \rightarrow a} f(x) = f(a) \) gefordert. \\
	Diese Definition unterscheidet sich in der nicht punktierten \(\varepsilon\)-Umgebung und es gilt \(f(a)\) anstatt b.
	\item \(f\) ist stetig auf \(A\), wenn \(f\) in jedem Punkt \(a \in A\) stetig ist.
\end{enumerate}

\subsection{Bemerkung}
\begin{enumerate}[(i)]
	\item Kompositionen stetiger Funktionen sind wieder stetig: \(f, g\) stetig \(\Rightarrow f+g, f-g, ...\) stetig
	\item Das Folgenkriterium überträgt sich: \\
	Sei \((a_k)_{k=1}^\infty \) eine Folge in \(A\) mit \(\lim_{k \rightarrow \infty} a_k = a  \quad \Leftrightarrow \quad \lim_{k \rightarrow \infty} f(a_k) = f(a)\)
	\item Ist \(A\) kompakt, dann nimmt eine stetige Funktion \(f : A \rightarrow \R\) immer ein Maximum und Minimum an:
	\begin{align*}
		\exists~ x_m, x_M \in A \mitt f(x_m) = \min_{x \in A} f(x), f(x_M) = \max_{x \in A} f(x)
	\end{align*}
\end{enumerate}

\section{Partielle Ableitungen, Richtungsableitungen}
\subsection{Definition der partiellen Ableitung}
Die Funktion \(f : A \subset \Rn \rightarrow \Rm\) heißt \textbf{partielle differenzierbar} in \(a \in A\) nach der \(k\)-ten Variable \(x_k \mitt k \in \left\{ 1, ..., n\right\} \) wenn der folgender Grenzwert existiert:
\begin{align*}
	\frac{\partial}{\partial x_k} f(a) = f_{x_k}(a) = \lim_{h \rightarrow 0} \frac{f(a + h \cdot e_k) - f(a)}{h}
\end{align*}
Existieren alle partielle Ableitungen \(f_{x_1}(a), ..., f_{x_n}(a)\), dann ist der \textbf{Gradient} von \(f\) wie folgt definiert:
\begin{align*}
	\nabla f(a) = \left( \begin{array}{c}
		f_{x_1} (a) \\
		\vdots \\
		f_{x_n} (a)
	\end{array} \right)
\end{align*}
und die Funktion \(f\) heißt mindestens einmal partielle differenzierbar. Sind die partiellen Ableitungen \(f_{x_1}(a), ..., f_{x_n}(a)\) zudem stetig, so heißt \(f\) einmal stetig differenzierbar: \( f \in C^1(A,\Rm)\) oder kurz \( f \in C^1(A) \).

\subsection{Beispiel}
Sei \(f(x,y,z) = x^2 - xy + 3z\)
\begin{align*}
	\frac{\partial}{\partial x} f(x,y,z) &= 
	\lim_{h \rightarrow 0} \frac{f(x+h,y,z) - f(x,y,z)}{h} \\
	&= \lim_{h \rightarrow 0} \frac{(x+h)^2 - (x+h)y + 3z - ( x^2 - xy + 3z)}{h} \\
	&= \lim_{h \rightarrow 0} \frac{(x+h)^2 - x^2}{h} - \frac{(x+h)y-xy}{h} + \frac{3z - 3z}{h} \\
	&= \left( \frac{d}{dx} x^2 \right) - \left( \frac{d}{dx} x \right)y + \left( \frac{d}{dx} 0 \right) z \\
	&= 2x - y + 0 \\
	&\Rightarrow \nabla f(x,y,z) = \left( \begin{array}{c}
		2x - y \\
		-x \\
		3	
	\end{array} \right)
\end{align*}

\subsection{Definition der Richtungsableitung}
Sei \(a, r \in \Rn\) mit \(\Vert r \Vert = 1\) (normiert), \(f: \Rn \rightarrow \Rm\), dann heißt der folgende Grenzwert die Richtungsableitung von \(f\) bei \(a\) in Richtung \(r\):
\begin{align*}
	\frac{\partial}{\partial r} f(a) = f_r (a) = \lim_{h \rightarrow 0} \frac{f(a + h \cdot r) - f(a)}{h}
\end{align*}

\subsubsection*{Bemerkung}
\begin{enumerate}[(i)]
	\item Ist \(r = e_k\), dann erhalten wir gerade eine partielle Ableitung.
	\item Es gibt Funktionen die in \(a\) in \underline{jede Richtung differenzierbar} sind, aber in \(a\) \underline{nicht stetig} sind!
\end{enumerate}

\section{Total Differenzierbarkeit}
\textit{Idee}: Differenzierbare Funktionen sind lokal im Punkt \(x_0\) linear approximierbar:
\begin{align*}
	f(x) = f(x_0) + f'(x_0)(x-x_0) + \underbrace{r(x)\Vert x-x_0 \Vert}_{\tilde{r}(x)}
\end{align*}
Dabei muss der Fehler \(\tilde{r}(x) = r(x)\Vert x - x_0 \Vert \) \textit{schneller gegen Null gehen als \(x\) gegen \(x_0\)} also muss \(\tilde{r}(x) = \hbox{o}(x - x_0)\) gelten (Landau-Notation: klein-oh).

\subsection{Definition der totalen Differenzierbarkeit}
Sei \(f : A \subset \Rn \rightarrow \Rm, A\) offen, \(a \in A\)
\begin{enumerate}[(i)]
	\item Die Funktion \(f\) nennt man \textbf{total differenzierbar} bei \(a\), wenn eine Matrix \(A \in \Rmxn\) existiert, mit der sich die Funktion \(f\) in einer \(\varepsilon\)-Umgebung um \(a\) mittels einer Hyperebene approximieren lässt:
	\begin{align*}
		f(x) = f(a) + A(x-a) + r(x)\Vert x - a \Vert
	\end{align*}
	Dann nennt man die Matrix \(A = f'(a) = \frac{\partial}{\partial x} f(a)\) die total Ableitung von \(f\) in \(a\).
	\item Ist \(f = \left( \begin{array}{c} f_1 \\ \vdots \\ f_m \end{array} \right) \) partiell diff'bar, so nennt man die Ableitung \textbf{Jacobi-Matrix}:
	\begin{align*}
		f'(a) = \frac{\partial}{\partial x} f(a) = J_f(a) = \left( \begin{array}{ccc}
			\frac{\partial}{\partial x_1} f_1(a) & \hdots & \frac{\partial}{\partial x_n} f_1(a) \\
			\vdots & & \vdots \\
			\frac{\partial}{\partial x_1} f_m(a) & \hdots & \frac{\partial}{\partial x_n} f_m(a) \\
		\end{array} \right) \in \Rmxn
	\end{align*}
	\textit{Bemerkung}: Es gilt: \( \exists f'(a) \quad \Rightarrow \quad f'(a) = J_f(a) \), \underline{nicht} aber die Gegenrichtung! Es kann also sein, dass die Jacobi-Matrix \(J_f\) existiert die Funktion aber \underline{nicht total diff'bar} ist.
\end{enumerate}

\subsection{Beispiele}
\begin{enumerate}[(i)]
	\item \begin{align*}
		f(r,\varphi) = r \cdot \left( \begin{array}{c} \cos \varphi \\ \sin \varphi \end{array} \right) \quad \Rightarrow \quad J_f = \left( \begin{array}{cc}
			\cos \varphi & -r \sin \varphi \\
			\sin \varphi & r \cos \varphi
		\end{array} \right)
	\end{align*}
	\item \(f(x) = a + b^\top(x - x_0), \quad f : \Rn \rightarrow \R, \quad a \in \R, \quad b,x_0 \in \Rn\) \\
	\(\Rightarrow \quad f(x_0) = a, \quad f'(x_0) = b^\top \)
	\item \(f(x) = a + A(x - x_0), \quad f : \Rn \rightarrow \Rm, \quad a \in \Rm, \quad A \in \Rmxn, \quad x_0 \in \Rn\) \\
	\(\Rightarrow \quad f(x_0) = a, \quad f'(x_0) = A \)
\end{enumerate}
\textit{Bemerkung}: Beispiel (ii) und (iii) sind lineare Funktionen.

\subsection{Satz}
Ist \(f : A \subset \Rn \rightarrow \Rm\) in jedem Punkt \(a \in A\) total differenzierbar, so ist \(f\) stetig in \(A\).
\textit{Beweis}:
\begin{align*}
	f(x) &= \underbrace{f(a)}_{\stackrel{x \rightarrow a}{\rightarrow} f(a)} + 
	\underbrace{A \underbrace{(x-a)}_{\stackrel{x \rightarrow a}{\rightarrow} 0 \in \Rn}}_{\stackrel{x \rightarrow a}{\rightarrow} 0 \in \Rn} + 
	\underbrace{r(x)}_{\stackrel{x \rightarrow a}{\rightarrow} 0 \in \Rm} 
	\underbrace{\Vert x - a \Vert}_{\stackrel{x \rightarrow a}{\rightarrow} 0 \in \R}
	\qquad \mitt r(x) \stackrel{x \rightarrow a}{\rightarrow} 0 \\
	\lim_{x \rightarrow a} f(x) &= f(a) \quad _\square\\
\end{align*}

\subsection{Satz}
Sei \(f : A \subset \Rn \rightarrow \Rm, a \in A\)
\begin{enumerate}[a.~]
	\item Ist \(f\) total differenzierbar in \(a\), so gilt
	\begin{enumerate}[(i)]
		\item \(f'(a) = J_f(a) \)
		\item \(f\) ist in jede Richtung \(r\) differenzierbar mit: \( \frac{\partial}{\partial r} f(a) = J_f(a) \cdot r \)
	\end{enumerate}
	\item Existieren in \(a\) alle partiellen Ableitungen (also \underline{alle Komponenten der Jacobi-Matrix}) und diese \underline{stetig} sind \(\quad \Rightarrow \quad f\) ist in \(a\) total differenzierbar.
\end{enumerate}



\end{document}








