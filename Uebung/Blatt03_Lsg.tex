\documentclass[11pt,a4paper]{article} 
\usepackage[utf8]{inputenc}
\usepackage[ngerman]{babel}
\usepackage{amsmath}
\usepackage{amssymb}
\usepackage{bbm}
\usepackage{geometry}
\usepackage{graphicx}
\usepackage{verbatim}
\geometry{
  left=3cm,
  right=3cm,
  top=3cm,
  bottom=3cm,
  bindingoffset=0mm
}
\usepackage{enumerate}
\usepackage[T1]{fontenc} 
\usepackage{ucs}
\usepackage[utf8]{inputenc}
\usepackage{color}

\newcommand {\Q}	{\mathbb{Q}}
\newcommand {\R}	{\mathbb{R}}
\newcommand {\C}	{\mathbb{C}}
\newcommand {\Rn}	{\mathbb{R}^n}
\newcommand {\Rm}	{\mathbb{R}^m}
\newcommand {\Rzwei}	{\mathbb{R}^2}
\newcommand {\Rnxn}	{\mathbb{R}^{n \times n}}
\newcommand {\Rmxn}	{\mathbb{R}^{m \times n}}
\newcommand {\Rmxm}	{\mathbb{R}^{m \times m}}
\newcommand {\Cn}	{\mathbb{C}^n}
\newcommand {\Cnxn}	{\mathbb{C}^{n \times n}}
\newcommand {\N}	{\mathbb{N}}
\newcommand{\1}    	{\mathbbm{1}}
\newcommand{\Onot}		{\mathcal{O}}
\newcommand{\diag}	{\textrm{diag}}
\newcommand{\mitt}	{\textrm{ mit }}

\author{Ruedi Lüthi}
\title{Blatt 2}

\begin{document}

	\maketitle
	
	\section*{14}
	Sei \(f : A \subset \Rn \rightarrow \Rm, a \in A\) und sei \( \xi = \left( \begin{array}{c}
			\xi_1 \\ \vdots \\ \xi_n
		\end{array} \right) \), damit
	\begin{align*}
		&z^{(k)} = a + \sum_{j=1}^k \xi_i \cdot e_i 
		\quad\Rightarrow\quad z^{(n)} = a + \xi = b \\
		&z^{(k)} - z^{(k-1)} = a + \sum_{j=1}^{k} \xi_i e_i - 
		\left( a + \sum_{j=1}^{k-1} \xi_i e_i \right) = \xi^k e_k
	\end{align*}
	Es gilt, nach dem Mittelwertsatz der Differentialrechnung:
	\begin{align*}
		\exists~ \eta \in (0,1) \quad\mitt\quad f(b) = f(a) + \nabla f\left(a + \eta(b-a)\right)^\top(b-a)
	\end{align*}
	einsetzen:
	\begin{align*}
		f(z^{(k)}) &= f(z^{(k-1)}) + \nabla f\left(z^{(k-1)} + \eta(z^{(k)} - z^{(k-1)})\right)^\top(z^{(k)} - z^{(k-1)}) \\
		&= f(z^{(k-1)}) + \frac{\partial}{\partial z_k} 
		f\left(
			\underbrace{ z^{(k-1)} + \eta \cdot \xi_k e_k}
			_{= y^{(k)}}
		\right)^\top \xi_k e_k
	\end{align*}
	aufsummiert über alle \(k \in \{1, ..., n\}\):
	\begin{align*}
		f(b) = f(a+\xi) &= f(a) + \sum_{k=1}^n \frac{\partial}{\partial z_k} f\left(y^{(k)}\right) \xi_k \\
		&= f(a) + \sum_{k=1}^n \frac{\partial}{\partial z_k} f\left(y^{(k)}\right) \xi_k + 
		\underbrace{
			\sum_{k=1}^n \frac{\partial}{\partial z_k} f(a) \xi_k - 
			\sum_{k=1}^n \frac{\partial}{\partial z_k} f(a) \xi_k
		}_{=0} \\
		&= f(a) + 
		\underbrace{
			\sum_{k=1}^n \frac{\partial}{\partial z_k} f(a) \xi_k
		}_{
			=~ \nabla f(a) \xi =~ f'(a)(b-a)
		}
		+
		\underbrace{
			\sum_{k=1}^n \left(
				\frac{\partial}{\partial z_k} f\left(y^{(k)}\right) -
				\frac{\partial}{\partial z_k} f(a)
			\right) \xi_k
		}_{
			=~ \tilde{r}(\xi) \quad \textrm{(Restglied)}
		}
	\end{align*}
	Aufgrund der Stetigkeit folgt:
	\begin{align*}
		y^{(k)} \stackrel{\xi \rightarrow 0}{\rightarrow} a \textrm{ und }
		\nabla f \textrm{ stetig} & \quad \Rightarrow \quad
		\nabla f(y^{(k)}) \stackrel{\xi \rightarrow 0}{\rightarrow} \nabla f(a) \\
		 & \quad \Rightarrow \quad \lim_{\xi \rightarrow 0} \frac{\tilde{r}(\xi)}{\Vert \xi \Vert} = 0
	\end{align*}
		
\end{document}