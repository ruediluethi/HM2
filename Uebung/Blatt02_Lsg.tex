\documentclass[11pt,a4paper]{article} 
\usepackage[utf8]{inputenc}
\usepackage[ngerman]{babel}
\usepackage{amsmath}
\usepackage{amssymb}
\usepackage{bbm}
\usepackage{geometry}
\usepackage{graphicx}
\usepackage{verbatim}
\geometry{
  left=3cm,
  right=3cm,
  top=3cm,
  bottom=3cm,
  bindingoffset=5mm
}
\usepackage{enumerate}
\usepackage[T1]{fontenc} 
\usepackage{ucs}
\usepackage[utf8]{inputenc}
\usepackage{color}

\newcommand {\Q}	{\mathbb{Q}}
\newcommand {\R}	{\mathbb{R}}
\newcommand {\C}	{\mathbb{C}}
\newcommand {\Rn}	{\mathbb{R}^n}
\newcommand {\Rzwei}	{\mathbb{R}^2}
\newcommand {\Rnxn}	{\mathbb{R}^{n \times n}}
\newcommand {\Rmxn}	{\mathbb{R}^{m \times n}}
\newcommand {\Rmxm}	{\mathbb{R}^{m \times m}}
\newcommand {\Cn}	{\mathbb{C}^n}
\newcommand {\Cnxn}	{\mathbb{C}^{n \times n}}
\newcommand {\N}	{\mathbb{N}}
\newcommand{\1}    	{\mathbbm{1}}
\newcommand{\Onot}		{\mathcal{O}}
\newcommand{\diag}	{\textrm{diag}}
\newcommand{\mitt}	{\textrm{ mit }}

\author{Ruedi Lüthi}
\title{Blatt 2}

\begin{document}

	\maketitle
	
	\section*{5}
	
	Sei \(f(x,y,z) = 4 \cos z \arctan \left( y e^{-x^2} \right) \)	\\
	
	\noindent
	Mit Definition:
	\begin{align*}
		&\lim_{h \rightarrow 0} \frac{f(x_0 + h r) - f(x_0)}{h} =\\
		&\lim_{h \rightarrow 0} \frac{4 \cos\left( \frac{\pi}{4} + h \sqrt{14} \right) \arctan \left( (1+h) e^{-(0+h)^2} \right) - 4 \cos\left( \frac{\pi}{4} \right) \arctan \left( 1 e^{-0^2} \right) }{h} = \\
		&\lim_{h \rightarrow 0}
		\frac{4 \cos\left( \frac{\pi}{4} + h \sqrt{14} \right)
		\arctan \left( (1+h) e^{-h^2}\right) - 4 \frac{1}{\sqrt{2}} \frac{\pi}{4} }{h} \stackrel{\textrm{Regel von de l'Hospital}}{=} \\
		&\lim_{h \rightarrow 0}
		\frac{
			-4\sin\left( \frac{\pi}{4} + h \sqrt{14} \right) \sqrt{14}  \arctan\left( (1+h)e^{-h^2} \right)
			+4\cos\left( \frac{\pi}{4} + h \sqrt{14} \right) \frac{e^{h^2}(1-2h-2h^2)}{e^{2h^2}+(1+h)^2}
		}{1} \\
		&-4 \frac{1}{\sqrt{2}}\sqrt{14}\frac{\pi}{4}
		+ 4 \frac{1}{\sqrt{2}}\frac{1}{2} = -\sqrt{7}\pi + \sqrt{2}
	\end{align*} \\
	
	\noindent
	Mit Jacobi-Matrix:
	\begin{align*}
		\nabla f (x,y,z) &= \left( \begin{array}{c}
			4 \cos z \frac{1}{1 + \left( ye^{-x^2} \right)^2}	\cdot y e^{-x^2} \cdot -2x = 
			-8 \cos z \cdot xy \frac{e^{-x^2}}{e^{-2x^2} \left( e^{2x^2} + y^2 \right)} = 
			- \frac{8 \cos z \cdot xy \cdot e^{x^2}}{e^{2x^2} + y^2} \\
			4 \cos z \frac{1}{1 + \left( ye^{-x^2} \right)^2} \cdot e^{-x^2} = 
			4 \cos z \frac{e^{-x^2}}{e^{-2x^2} \left( e^{2x^2} + y^2 \right)} = 
			\frac{4 \cos z \cdot e^{x^2} }{e^{2x^2} + y^2}\\
			- 4 \sin z \arctan \left( y e^{-x^2} \right)
		\end{array} \right) \\
		\nabla f\left( x_0 \right) &= 
		\nabla f\left( 0,1,\frac{\pi}{4} \right) = \left( \begin{array}{c}
			0 \\
			\frac{\frac{4}{\sqrt{2}}}{1 + 1} = \frac{2}{\sqrt{2}} \\
			-4 \frac{1}{\sqrt{2}} \frac{\pi}{4} = - \frac{\pi}{\sqrt{2}}
		\end{array} \right) \\
		\nabla f\left( x_0 \right)^\top r &= \left( \begin{array}{c}
			0 \\
			\frac{2}{\sqrt{2}} \\
			- \frac{\pi}{\sqrt{2}}
		\end{array} \right)^\top \left( \begin{array}{c}
			1 \\ 1 \\ \sqrt{14}
		\end{array} \right) = 0 + \frac{2}{\sqrt{2}} - \frac{\pi \sqrt{14}}{\sqrt{2}} = \sqrt{2} - \pi \sqrt{7}
	\end{align*}
	
	\section*{6}
	\color{red}
	Warum gilt \(\vert y \vert < \delta\)? 
	\color{black} \\
	Sei \(\varepsilon > 0\) beliebig und \(\delta_\varepsilon = \varepsilon\), dann gilt für \(f(x,y)\) stetig:
	\begin{align*}
		\left\vert f(x,y) - f(0,0) \right\vert = 
		\left\vert \frac{\vert x \vert y}{\vert x \vert + y^2} - 0\right\vert \stackrel{\vert x \vert \neq 0}{=} 
		\left\vert \frac{y}{1 + \frac{y^2}{\vert x \vert}} \right\vert \stackrel{\frac{y^2}{\vert x \vert} \geqslant 0}{\leqslant}
		\vert y \vert < \delta_\varepsilon = \varepsilon
	\end{align*}
	
	\noindent
	Sei \(r = (r_x, r_y)^\top\) eine beliebige Richtung:
	\begin{align*}
		\frac{\partial}{\partial r} f(x_0) &= 
		\lim_{h \rightarrow 0} \frac{f(x_0 + h r) - f(x_0)}{h} =
		\lim_{h \rightarrow 0} \frac{f(0 + h r_x, 0 + h r_y) - 0}{h} \\
		&= \lim_{h \rightarrow 0} \frac{\frac{\vert h r_x \vert h r_y}{\vert h r_x \vert + \left( h r_y \right)^2}}{h} =
		\lim_{h \rightarrow 0} \frac{\vert h r_x \vert  r_y}{\vert h r_x \vert + h^2 r_y^2} =
		\lim_{h \rightarrow 0} \frac{\vert h r_x \vert r_y}{\vert h r_x \vert ( 1 + \underbrace{\frac{h^2 r_y^2}{\vert h r_x \vert}}_{\stackrel{h \rightarrow 0}{\rightarrow} 0})} = r_y
	\end{align*}
	
	\noindent
	Die Funktion ist nicht total differenzierbar da die Jacobi-Matrix nicht existiert, denn:
	\begin{align*}
		\frac{\partial}{\partial x} \frac{\vert x \vert y}{\vert x \vert + y^2} \quad \Rightarrow \quad \textrm{ nicht diff'bar da } \vert x \vert \textrm{ nicht diff'bar in } x = 0 \textrm{ ist}
	\end{align*}
	
	\section*{7}	
	\begin{align*}
		f(x,y) = \left\{ \begin{array}{ll}
			0 & (x,y) = (0,0) \\
			\frac{y \sin(xy)}{x^2 + y^4} & (x,y) \neq (0,0)
		\end{array} \right)
	\end{align*}	
	
	Nebenrechnung: Für \(\alpha \in (0,\frac{\pi}{2}) \) gilt:
	\begin{align*}
		\frac{2 \alpha}{\pi} < \sin(\alpha) < 1
	\end{align*}
	
	Nicht stetig, da
	\begin{align*}
		x_0(t) = \left( \begin{array}{c}
			t \\ \sqrt{t}
		\end{array}	\right) \quad
		\lim_{t \rightarrow 0} f(x_0(t)) = 
		\lim_{t \rightarrow 0} \frac{\sqrt{t} \sin(t \sqrt{t})}{t^2 + \sqrt{t}^4} > \lim_{t \rightarrow 0} \frac{\sqrt{t} \frac{2}{\pi}t\sqrt{t}}{t^2 + t^2} = \frac{2}{\pi} \frac{t^2}{2t^2} = \pi
	\end{align*}		
	
	Sei \(r = (r_x, r_y)^\top\) eine beliebige Richtung:
	\begin{align*}
		\frac{\partial}{\partial r} f(x_0) &= 
		\lim_{h \rightarrow 0} \frac{f(x_0 + h r) - f(x_0)}{h} =
		\lim_{h \rightarrow 0} \frac{f(0 + h r_x, 0 + h r_y) - 0}{h} \\
		&= \lim_{h \rightarrow 0} \frac{\frac{h r_y \sin(h^2 r_x r_y)}{h^2 r_x^2 + h^4 r_y^4}}{h} = 
		\lim_{h \rightarrow 0} \frac{r_y \sin(h^2 r_x r_y)}{h^2 r_x^2 + h^4 r_y^4} \stackrel{\textrm{l'H}}{=} 
			\lim_{h \rightarrow 0} \frac{r_y \cos( h^2 r_x r_y) 2h r_x r_y}{2 h r_x^2 + 4h^3 r_y^4} \\
			&= \lim_{h \rightarrow 0} \frac{r_x r_y^2 \overbrace{\cos( h^2 r_x r_y)}^{\stackrel{h \rightarrow 0}{\rightarrow}1}}{r_x^2 + \underbrace{2h^2 r_y^4}_{\stackrel{h \rightarrow 0}{\rightarrow}0}} = \frac{r_y^2}{r_x}
	\end{align*}
	
	\section*{10}
	\subsection*{a)} Sei \((a_k) = \left( \begin{array}{c} (a_k)_1 \\ \vdots \\ (a_k)_n \end{array} \right) \subset \Rn\) eine Folge und sei \(a = \left( \begin{array}{c} a_1 \\ \vdots \\ a_n \end{array} \right) \in \Rn\).  
	\begin{align*}
		\textrm{zu zeigen: } \quad \lim_{k \rightarrow \infty} (a_k) = a \quad \Leftrightarrow \quad \lim_{k \rightarrow \infty} (a_k)_j = a_j \quad \forall~ j = 1, ..., n
	\end{align*}
	es gilt:
	\begin{align*}
		\vert (a_k)_j - a_j \vert = \sqrt{\vert (a_k)_j - a_j \vert^2} \leqslant 
		\underbrace{\sqrt{\vert (a_k)_1 - a_1 \vert^2 + ... + \vert (a_k)_n - a_n \vert^2}}_{= \Vert (a_k) - a \Vert_2}
		\leqslant \sqrt{n} \max_{j\in\{1,...,n\}} \left( (a_k)_j - a_j \right)
	\end{align*}
	Gilt \(\lim_{k \rightarrow \infty} \max_{j\in\{1,...,n\}} \left( (a_k)_j - a_j \right) \rightarrow 0\) so gehen auch alle andere Komponenten gegen Null.
	
	\subsection*{b)}
	Wegen \((a_k) \subset \Rn\) beschränkt gilt:
	\begin{align*}
		\exists~ c > 0 \mitt \Vert (a_k) \Vert < c \quad \forall k
	\end{align*}
	Weiter gilt:
	\begin{align*}
		c > \Vert (a_k) \Vert > \Vert \underbrace{(a_k) \cdot e_i}_{= (b_k)} \Vert \mitt (b_k) \subset (a_k)
	\end{align*}
	Aus der Konstruktion von \((b_k)\) folgt:
	\begin{align*}
		(b_k) = \left( \begin{array}{c}
			0 \\ \vdots \\ 0 \\
			(b_k)_i \\
			0 \\ \vdots \\ 0
		\end{array} \right)
	\end{align*}
	Dabei ist \((b_k)_i\) eine beschränkte Folge in \(\R\) und nach dem Satz von Bolzano-Weierstraß besitzt diese eine konvergente Teilfolge.
	
	
\end{document}